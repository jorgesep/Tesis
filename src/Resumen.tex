\resumen{
\addtocontents{toc}{\vspace{1em}}  % Add a gap in the Contents, for aesthetics

Este trabajo de tesis coloca a disposición de la comunidad de investigación en visión por computador, una pareja de herramientas de software que facilita, la integración, ejecución y posterior evaluación de desempeño, de algoritmos que realizan separación de siluetas de su imagen de fondo (\textit{Background Subtraction}). El sistema de software se divide en dos módulos principales; el primero crea una capa de abstracción (clases C++) para integrar los distintos algoritmos que desarrolla la comunidad de investigación, y proporciona un conjunto de interfaces que permiten una rápida y fácil integración con este sistema de software. Incorpora además, el algoritmo en estado de arte \textit{SAGMM} (\textit{Self-Adaptive Gaussian Mixture Model}) orientado a separación imágenes de fondo, basado en mixtura de componentes gaussiano \textit{GMM} (\textit{Gaussian Mixture Model}), y sirve de ejemplo para la integración de nuevos algoritmos en este sistema. El segundo módulo es un programa ejecutable que engloba un conjunto de métricas de evaluación de calidad, empleadas en las sistemas de clasificación y reconocimiento de patrones. Se detallan las mediciones estadísticas de rendimiento en clasificación binaria \textit{`sensitividad'} y \textit{`especificidad'}, asimismo, la métrica \textit{F-Measure} que es el promedio ponderado de \textit{Precision} y \textit{Recall}, o el coeficiente de correlación de Matthews \textit{MCC}. De manera que el ser utilizadas en su conjunto proporcionan una idea general del desempeño total de un algoritmo. El programa de evaluación de desempeño, funciona en base a la comparación de resultados, necesita de entrada un conjunto de imágenes resultantes (mascaras de siluetas), generadas por el algoritmo que se intenta evaluar y su correspondiente imagen de referencia. Este trabajo hace un uso extensivo de curva de operaciones características \textit{ROC}, para comparar rendimiento de distintos algoritmos y localizar un punto de operación óptimo de los algoritmos que se evalúan. Se realiza también la evaluación de un grupo de algoritmos basado en mixtura de componentes gaussianas, sobre el conjunto de datos MuHAVI, el resultado genera y publica el conjunto total de siluetas resultantes con el algoritmo estado de arte \textit{SAGMM} incluido en el primer modulo de software. El trabajo concluye con el análisis de las métricas más relevantes usadas en la evaluación de los algoritmo.

}